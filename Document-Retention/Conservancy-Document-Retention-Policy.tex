%% LyX 2.0.1 created this file.  For more info, see http://www.lyx.org/.
%% Do not edit unless you really know what you are doing.
\documentclass[english]{article}
\usepackage[T1]{fontenc}
\usepackage{babel}
\usepackage{xunicode}
\begin{document}

\title{Software Freedom Conservancy Document Retention Policy}

\maketitle

\section{Purpose}

Software Freedom Conservancy (``Conservancy'') has adopted a document
retention and destruction policy (``Policy'') in order to define
the record retention responsibilities of Conservancy officers, staff,
and board members for maintaining and documenting the storage and
destruction of Conservancy's documents and records.


\section{General Guidelines}

Records should not be kept if they are no longer needed for the operation
of the business or required by law. Unnecessary records should be
eliminated from the files. The cost of maintaining records is an expense
which can grow unreasonably if good housekeeping is not performed.
A mass of records also makes it more difficult to find pertinent records.

From time to time, Conservancy may establish retention or destruction
policies or schedules for additional categories of records in order
to ensure legal compliance, and also to accomplish other objectives,
such as cost management. Conservancy has identified several categories
of documents that warrant special consideration; these categories
are listed below. While Conservancy has established minimum retention
periods for these categories, the retention of these documents - and
of records not included in the identified categories - should be determined
primarily by the application of the General Guidelines affecting document
retention in this Section, as well as the exception for litigation-relevant
records described in the next Section and any other pertinent factors.


\section{Exception for GPL Compliance and Litigation-Relevant Documents}

Conservancy expects all staff, officers, board members, and PLCs to
comply fully with this Policy and any other written document retention
policies, with the following notable exception to any stated destruction
schedule: if you believe, or if Conservancy informs you, that certain
Conservancy and/or Project records relate to efforts to bring companies
into compliance with the General Public License, enforcing the GPL,
or in some way relates to litigation or potential litigation (i.e.,
a dispute that could result in litigation), then you \textbf{must}
preserve those records until it is determined that the records are
no longer needed. That exception supersedes any previously or subsequently
established destruction schedule for those records.


\section{Document Storage Methods}

All Conservancy documents are stored electronically in a version-controlled
repository on a remote server that is backed up to drives on our local
premises, with the following exceptions: 
\begin{itemize}
\item Conservancy stores all e-mails are stored on a separate e-mail server
\item Conservancy retains hard copies of all contracts executed with parties
outside of the United States for at least the full term of each contract
\end{itemize}

\section{Minimum Retention Periods for Conservancy Documents}


\subsection{Organizational Documents 	}

The following records should be retained permanently:
\begin{itemize}
\item Conservancy Articles of Incorporation
\item Conservancy By-laws
\item Conservancy's IRS Form 1023 - which is also to be made available for
public inspection upon request
\item Board meeting minutes
\end{itemize}

\subsection{Tax Records}

Tax records should be retained for at least seven years from the date
of filing the applicable return. These records include:
\begin{itemize}
\item IRS Form 990 and related schedules
\item Financial audit letter and supporting documents
\item Employment tax records
\item Documents concerning payroll, expenses, proof of donor contributions
\item Documented accounting procedures
\end{itemize}

\subsection{Financial Records}

In general, Conservancy's financial records should be kept for at
least seven years. These records include:
\begin{itemize}
\item Accounts payable ledgers and schedules
\item Accounts receivable ledgers and schedules
\item Bank reconciliations, bank statements, deposit slips and checks (except
for as stated below)
\item Donation records
\item Expense reimbursement requests and supporting receipts and documentation
\end{itemize}
Certain important financial records are to be kept permanently. These
records include: 
\begin{itemize}
\item Checks and/or proofs of payment for expenditures over US\$100,000
\end{itemize}

\subsection{Employment/Personnel Records}

All employment and personnel records should be kept for at least seven
years, with the following exceptions:
\begin{itemize}
\item IRS Form 990 and related schedules: at least seven years from the
date of filing the applicable return
\item Financial audit letter and supporting documents: at least seven years
from the date of filing the applicable return
\item Employment applications and related documents (including correspondence
and CVs from prospective candidates): at least three years
\item Employee offer letters, and confirmations of acceptance: kept permanently
\end{itemize}

\subsection{Other Corporate Records}

Other corporate records shall be kept according to the schedule below:
\begin{itemize}
\item Insurance policies (both current and expired), claims, and related
reports: retained permanently
\item Press releases: retained permanently
\item Trademark and copyright applications and registration materials: retained
permanently
\item Executed fiscal sponsorship agreements: retained permanently
\item Internal strategic documents: at least three years
\item Correspondence from Member Project liaisons re: Member Project matters,
including disbursement requests: at least seven years
\end{itemize}

\subsection{Correspondence}
\begin{itemize}
\item Except as classified above, letters received in hard copy shall be
kept for at least one year.
\item Except as classified above, e-mail correspondence in Conservancy e-mail
accounts shall be archived for at least one year, with the exception
of the correspondence sent to the ``compliance@sfconservancy.org''
account, which shall be archived for at least seven years. 
\end{itemize}

\subsection{Reports from Projects}


\subsection{Blogs and Press Releases}

Conservancy should retain permanent copies of all official Conservancy
blogs, press releases and 


\subsection{Contracts}

Except as classified above, all contracts entered into by Conservancy
should be retained for at least seven years from the date of filing
the applicable return. These records include:
\begin{itemize}
\item IRS Form 990 and related schedules
\item Financial audit letter and supporting documents
\item Documents concerning payroll, expenses, proof of donor contributions
\item Documented accounting procedures
\end{itemize}

\subsection{Software Development Contract Deliverables}

All code and supporting documentation written by a developer in fulfillment
of a contract with Conservancy should be archived electronically for
at least three years. 


\subsection{Presentations}

All presentations, including slides, charts, and supporting visual
aids, prepared for use by a Conservancy officer or staff member shall
be kept for at least two years. 


\subsection{Drafts}

Notwithstanding the above, once the final copy of a document has been
completed, the drafts may be recycled or deleted, unless they are
documents of legal value. For documents determined to be of legal
value, drafts containing comments shall be saved for at least two
years, and drafts without comment may be destroyed once the final
version is complete. 


\section{Destruction of Documents}

Conservancy reserves the right to destroy hard copies of documents
by shredding or fire after the expiration of the applicable document
retention schedule. Conservancy reserves the right to destroy electronic
copies of documents by fire or other proven means to destroy such
media after the expiration of the applicable document retention schedule.

All permitted document destruction shall be halted if the organization
is being investigated by a governmental law enforcement agency, and
routine destruction shall not be resumed without the written approval
of Conservancy's General Counsel or Executive Director. No documents
are to be concealed, altered, or destroyed with the intent to obstruct
a legal investigation or litigation. 
\end{document}
